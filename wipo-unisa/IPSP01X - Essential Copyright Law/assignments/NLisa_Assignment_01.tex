% Created 2017-03-18 Sat 17:36
% Intended LaTeX compiler: pdflatex
\documentclass[11pt]{article}
\usepackage[utf8]{inputenc}
\usepackage[T1]{fontenc}
\usepackage{graphicx}
\usepackage{grffile}
\usepackage{longtable}
\usepackage{wrapfig}
\usepackage{rotating}
\usepackage[normalem]{ulem}
\usepackage{amsmath}
\usepackage{textcomp}
\usepackage{amssymb}
\usepackage{capt-of}
\usepackage{hyperref}
\usepackage[margin=1.0in]{geometry}
\usepackage[backend=bibtex, style=ieee]{biblatex}
\addbibresource{/home/nuk3/course/training/csir/novellasers/bibliography/bibliography.bib}
\DeclareFieldFormat[inproceedings]{citetitle}{\textit{#1}}
\DeclareFieldFormat[inproceedings]{title}{\textit{#1}}
\DeclareFieldFormat[misc]{citetitle}{#1}
\DeclareFieldFormat[misc]{title}{#1}
\renewcommand*{\bibpagespunct}{%
\ifentrytype{inproceedings}
{\addspace}
{\addcomma\space}}
\AtEveryCitekey{\ifuseauthor{}{\clearname{author}}}
\AtEveryBibitem{\ifuseauthor{}{\clearname{author}}}
\author{Nyameko Lisa}
\date{\today}
\title{IPSP01X - Essential Copyright Law\\\medskip
\large Assignment 1 - Unique Number: 659072}
\hypersetup{
 pdfauthor={Nyameko Lisa},
 pdftitle={IPSP01X - Essential Copyright Law},
 pdfkeywords={},
 pdfsubject={},
 pdfcreator={Emacs 26.0.50 (Org mode 9.0.5)},
 pdflang={English}}
\begin{document}

\maketitle
\section*{Declaration}
\label{sec:org534d01f}
\begin{itemize}
\item I know that plagiarism is to use someone else’s work and pass it off as my own.
\item I know that plagiarism is wrong.
\item I confirm that this assignment is my own work.
\item I have acknowledged in the bibliography accompanying the assignment all the sources that I have used.
\item I have not directly copied without acknowledgement anything from the Internet or from any other source.
\item I have indicated every quotation and citation in a footnote or bracket linked to that quotation.
\item I have not allowed anyone else to copy my work and to pass it off as their own work.
\item I understand that if any unacknowledged copying whatsoever appears in my assignment I will receive zero per cent for the assignment.
\item I am aware of the UNISA policy on plagiarism and understand that disciplinary proceedings can be instituted against me by UNISA if I contravene this policy.
\item I indicate my understanding and acceptance of this declaration by
entering my name hereunder:
\begin{itemize}
\item Name: \textbf{Nyameko Lisa} (Student Number: \textbf{7874-909-3})
\end{itemize}
\end{itemize}
\newpage
\section{Who are the authors of English study guide? \textbf{[4]}}
\label{sec:org0f8e1c7}

In accordance with the provisions of section 1 of the
\citetitle{rsa78_copyrightact} of \citedate{rsa78_copyrightact}
\cite{rsa78_copyrightact} and as per article 2(1) of the Berne
Convention \cite{wipo86_berne}, the English study guide can be
considered as a piece of \textbf{literary work}, which is a \textbf{work of joint
authorship} and was first made or created by the ABC123 lectures, all
of whom enjoy \emph{`joint authorship'}.\\

In the case of \emph{`joint authorship'}, any one of the authors must be a
\textbf{qualified person} at the time the work or a substantial part thereof
was created. Assuming that as university employees, at least one of
the ABC123 lecturers are a citizen of, or domiciled in or permanent
resident of Berne Convention country [section 3, section
37]\cite{wipo86_berne}, it follows that the co-authors of the English
study guide are the ABC123 lectures, where the term \textbf{author} is used
as defined in
[section 1]\cite{rsa78_copyrightact}.

\section{Who is the copyright owner of the English study guide? \textbf{[3]}}
\label{sec:org62c1309}

As per [section 21(1)(d)]\cite{rsa78_copyrightact}, given that the
lecturers are all under employment contracts of service with the
university, and the fact that the study guide was produced\footnote{This is
not explicitly mentioned, but it is implied in the question, \emph{see
below.}} during the course of their employment with the university, it
follows that ownership of the copyright is held by the proprietor or
owner of the university or simply the university itself.\\

Alternatively however, specifically because we are not explicitly told
\emph{`when'} the authors prepared the study guide, - i.e. they could have
prepared a significant portion of the study guide before they were
under the employ of the university - it \emph{could} also be argued that
the ownership of the copyright belongs to the co-authors, as per
[section 21(1)(a)]\cite{rsa78_copyrightact}.

\section{What is the duration of the copyright in the English study guide? \textbf{[3]}}
\label{sec:org2b97381}

As per [section 3(2)(a), section 3(4)]\cite{rsa78_copyrightact}, the
duration of the copyright in the English study guide is the life of
all co-authors and fifty years from the end of the year in which the
last living author dies.

\section{Does Marli infringe the copyright in the English study guide? \textbf{[30]}}
\label{sec:orgf72d715}
As per the definitions provided in [section 1]\cite{rsa78_copyrightact},
copies of Marli's Afrikaans translation that are sold to prospective
students, constitute an \textbf{adaptation} of the \emph{`original'} \textbf{literary
work}, the English ABC123 study guide. Moreover, by those definitions,
given that Marli's Afrikaans translation is a reproduction of the
work, it will be considered a \textbf{copy}, and for the purposes of the
following arguments it will be referred to as the \textbf{infringing copy}.\\


As per [section 23(1), section 23(2)(b)]\cite{rsa78_copyrightact}, Marli
infringed on the copyright of the English study guide, when she
\begin{itemize}
\item made the Afrikaans translation without owning the copyright, or
being granted a licence by the university, or being authorized by
the university to make an \textbf{adaptation} of said study guide,
\item reproduced and distributed \emph{`infringing copies'} of said \textbf{adaptation} of
the work,
\item each time she exploited the \emph{`infringing copies'} it was a violation
of the copyright enjoyed in the original work.
\end{itemize}

causal connection \cite{burger85_bosal_afrika_v_grapnel} access
\cite{corbett89_galago_v_erasmus}

Even though Marli's translation itself enjoys copyright protection as
an \emph{`alteration'} of a \textbf{literary work}, as per [article
2]\cite{wipo86_berne}, every time the work is reproduced or distributed,
it will be an infringement of the copyright enjoyed by the English
study guide. Thus Marli is liable for copyright infringement whenever
she tries to exploit the translation, or \textbf{adaptation} of the work, as
stipulated in [article 6]\cite{wto17_trips}.

\subsection{Copyright Exceptions}
\label{sec:orgdf70414}
It is important to note that Marli's translation of the study guide
from English to Afrikaans, for her own personal use does not
constitute an unlawful act and copyright of the study guide was not
infringed upon, as per [section 12(1)(a)]\cite{rsa78_copyrightact}.
\subsection{Exclusive Rights}
\label{sec:org2e9b2ca}
However, as per [section 6(f)]\cite{rsa78_copyrightact}, the
\emph{`infringing copy'} violates the exclusive rights of the copyright
owner, specifically in respect of the right to either carry out or
authorize the \textbf{adaptation} of the English study guide.\\

The right to translate the English study guide vests in the
university and thus the exclusive rights of the copy right holder are
infringed by the unauthorized adaptation. This is a \emph{`direct infringement'}.\\

Moreover, as per [section 6(g)]\cite{rsa78_copyrightact}, and its
references to paragraphs (a) and (e) of that section, the \emph{`infringing
copy'} constitutes unlawful reproduction and transmission in  a
diffusion service of the \textbf{adaptation} of the \emph{`original'} work. These
are \emph{`direct infringements'}.

\subsection{Direct Infringement}
\label{sec:org280f241}
By performing any of the acts that a copyright vests the exclusive
right to either perform or authorize the performance thereof, without
the authors' consent constitutes a \emph{`direct infringement'}, [section
6]\cite{rsa78_copyrightact}.

\subsection{Indirect Infringement}
\label{sec:org0875d2a}
Each time the \emph{`infringer'} Marli, consciously furthers the commission
of any acts that only the authors are permitted to do or authorize to
do, constitutes an \emph{`indirect infringement'}. Such as is the case with
the sale [Section 23(2)(b)]\cite{rsa78_copyrightact}, of her Afrikaans
translations of the English study guide.

\section{Can Marli claim copyright in her translation? \textbf{[10]}}
\label{sec:org75f1b8b}

Marli's translation must meet the \emph{`inherent'} and \emph{`formal'} or
\emph{`statutory'} requirements for copyright to subsist in her work.
\begin{itemize}
\item Inherent Requirements
\begin{itemize}
\item The requirement of material embodiment is clearly satisfied as the
work exists in material form.
\item As per [article 9(2)]\cite{wto17_trips} and
[article 2]\cite{wipo96_copyright_treaty}, it can be argued that the
Afrikaans translation is a \emph{`particular form of expression of
thought'}, thus satisfying the requirement of originality.
\end{itemize}
\item Formal Requirements
\begin{itemize}
\item As a student of the university, it is implied that she is either a
citizen of, or domiciled in, or a permanent resident of a Berne
Convention country [section 3, section 37]\cite{wipo86_berne},
meaning that she was a \textbf{qualified person} at the time of creation
of the work and thus making her the author.
\item Moreover it is implied that the work was first made in South
Africa, [section 4]\cite{rsa78_copyrightact}, hence Marli may allege
that the translation meets the statutory requirements for the
subsistence of copyright in the work.
\end{itemize}
\end{itemize}

Given that translations can be considered as \emph{`alterations'} of
\textbf{literary works} and thus also enjoy copyright protection as per
[article 2]\cite{wipo86_berne}. Therefore Marli's Afrikaans translation
will enjoy copyright protection independently of the original English
study guide.\\

Moreover as per [section 2(3)]\cite{rsa78_copyrightact}, Marli's
translation is not ineligible for copyright even though it's creation,
reproduction, distribution and sale amount to infringements of
copyright in the English translation.\\

However, it must be noted that since Marli's translation was made
without the consent of the ABC123 authors or a license from the
university, Marli is unable to exploit her translated work without
infringing the copyright enjoyed in the English study guide.
\printbibliography
\end{document}
