% Created 2017-05-19 Fri 23:38
% Intended LaTeX compiler: pdflatex
\documentclass[11pt]{article}
\usepackage[utf8]{inputenc}
\usepackage[T1]{fontenc}
\usepackage{graphicx}
\usepackage{grffile}
\usepackage{longtable}
\usepackage{wrapfig}
\usepackage{rotating}
\usepackage[normalem]{ulem}
\usepackage{amsmath}
\usepackage{textcomp}
\usepackage{amssymb}
\usepackage{capt-of}
\usepackage{hyperref}
\usepackage[margin=0.56in]{geometry}
\usepackage[backend=bibtex, style=ieee]{biblatex}
\addbibresource{/home/nuk3/course/training/csir/novellasers/bibliography/bibliography.bib}
\DeclareFieldFormat[InProceedings]{citetitle}{\textit{#1}}
\DeclareFieldFormat[inproceedings]{title}{\textit{#1}}
\DeclareFieldFormat[misc]{citetitle}{#1}
\DeclareFieldFormat[misc]{title}{#1}
\renewcommand*{\bibpagespunct}{%
\ifentrytype{inproceedings}
{\addspace}
{\addcomma\space}}
\AtEveryCitekey{\ifuseauthor{}{\clearname{author}}}
\AtEveryBibitem{\ifuseauthor{}{\clearname{author}}}
\author{Nyameko Lisa}
\date{\today}
\title{IPSP01X - Essential Copyright Law\\\medskip
\large Assignment 3 - Exam}
\hypersetup{
 pdfauthor={Nyameko Lisa},
 pdftitle={IPSP01X - Essential Copyright Law},
 pdfkeywords={},
 pdfsubject={},
 pdfcreator={Emacs 25.2.1 (Org mode 9.0.7)},
 pdflang={English}}
\begin{document}

\maketitle
\section*{Declaration}
\label{sec:orgc976d7a}
\begin{itemize}
\item I know that plagiarism is to use someone else’s work and pass it off as my own.
\item I know that plagiarism is wrong.
\item I confirm that this assignment is my own work.
\item I have acknowledged in the bibliography accompanying the assignment all the sources that I have used.
\item I have not directly copied without acknowledgement anything from the Internet or from any other source.
\item I have indicated every quotation and citation in a footnote or bracket linked to that quotation.
\item I have not allowed anyone else to copy my work and to pass it off as their own work.
\item I understand that if any unacknowledged copying whatsoever appears in my assignment I will receive zero per cent for the assignment.
\item I am aware of the UNISA policy on plagiarism and understand that disciplinary proceedings can be instituted against me by UNISA if I contravene this policy.
\item I indicate my understanding and acceptance of this declaration by
entering my name hereunder:
\begin{itemize}
\item Name: \textbf{Nyameko Lisa} (Student Number: \textbf{7874-909-3})
\end{itemize}
\end{itemize}

\subsection*{NOTE}
\label{sec:orgbbfa2d2}
Please note that footnotes will be denoted as \footnote{This is a footnote.} and will
appear at the bottom of the page.\\
References will be denoted by \cite{rsa78_copyrightact} and will appear at the end of the document.
\newpage


\section{Name three types of copyright "work" embodied in Mia's published articles. [5]}
\label{sec:org5b7c351}

As per [section 1]\cite{rsa78_copyrightact}, Mia's articles constitute \emph{literary
works} in the form of written reports, \emph{artistic works} in the form of photographs
and \emph{published editions} in the form of the published articles appearing in
Evergreen magazine, all of which are eligible for copyright as per [sections
2(1)(a, c \& h)]\cite{rsa78_copyrightact} respectively.

\section{Is it possible for copyright to subsist in Roxy's drawing even if it infringes the copyright in another dress which she saw in a teen magazine? [20]}
\label{sec:org23678ff}
As per the definitions in [section 1 \& section 1(1)(c)]\cite{rsa78_copyrightact},
Roxy's drawing can be considered as an \textbf{artistic work}, moreover depending on
the degree to which there are substantial features in common with the original
dress featured in the teen magazine, it may be considered as an \textbf{artistic
adaptation} or \textbf{artistic reproduction}. Since Roxy was the first person to make or
create the drawing, it follows that she is the \textbf{author} of the drawing as per [section 1]\cite{rsa78_copyrightact}.\\

Roxy's drawing must meet the \emph{`inherent'} and \emph{`formal'} or
\emph{`statutory'} requirements for copyright to subsist in her work.
\subsection{Inherent Requirements}
\label{sec:orge9bc30b}
\begin{itemize}
\item The requirement of material embodiment is clearly satisfied as the
work exists in material form.
\item As per [article 9(2)]\cite{wto17_trips} and
[article 2]\cite{wipo96_copyright_treaty}, it can be argued that the
Roxy's drawing is a \emph{`particular form of expression of
thought'}, thus satisfying the requirement of originality.
\end{itemize}
\subsection{Formal Requirements}
\label{sec:org425b6cc}
\begin{itemize}
\item As a student of a school in Cape Town, it is implied that she is either a
citizen of, or domiciled in, or a permanent resident of a Berne
Convention country [section 3, section 37]\cite{wipo86_berne},
meaning that she was a \textbf{qualified person} at the time of creation
of the work and thus making her the author.
\item Moreover it is implied that the work was first made in South
Africa, [section 4]\cite{rsa78_copyrightact}, hence Roxy may allege
that the drawing meets the statutory requirements for the
subsistence of copyright in the work.
\end{itemize}

\subsection{Yes, copyright subsists in Roxy's work.}
\label{sec:orgb39c1c2}
As an \emph{`adaptation of an artistic work'}, Roxy's drawing also enjoys copyright
protection, independently of the original copyright protected dress appearing in
the teen magazine, as per [article 2(3)]\cite{wipo86_berne}. By reproducing the
dress she saw in the teen magazine and / or making an adaptation thereof, Roxy
has infringed the exclusive rights of the authors / copyright owners to perform or
authorize those actions, as per [section 7(a),(e)]\cite{rsa78_copyrightact}.\\

Moreover as per [section 2(3)]\cite{rsa78_copyrightact}, Roxy's drawing is not
ineligible for copyright even though it's creation would
amount to infringements of copyright in the original dress appearing in the teen
magazine.\\

Lastly Roxy's drawing does not infringe on copyright if she made the drawing
purely for private and personal use, as per [section
12(1)(a)]\cite{rsa78_copyrightact}. However, she will be unable to exploit her
design without infringing on the copyright enjoyed in the dress appearing in the
teen magazine.

\section{When will the copyright in their wedding photographs expire? [5]}
\label{sec:org6093da4}
As per [section 3(2)(b)]\cite{rsa78_copyrightact}, the copyright will expire fifty
years from the end of the year in which Ann and Mark either publish or consent
to make their wedding photographs publicly available. If neither of the two
events have occurred within fifty years of their wedding, then the copyright
will expire in fifty years from the end of 2013, i.e. at the end of year 2063.

\section{Does Raymond infringe on the copyright in Amanda's paintings? [25]}
\label{sec:org68c9e64}
As per the definitions provided in [section 1]\cite{rsa78_copyrightact},
Raymond's paintings, constitute \textbf{adaptations} of the \textbf{infringing copies} of
Amanda's \emph{`original'} \textbf{artistic works}, i.e. her paintings, the photographs
of her paintings and the published editions of the photographs of her
paintings.\\

As per definitions in [section 1]\cite{rsa78_copyrightact} and [section
2(1)(c),(h)]\cite{rsa78_copyrightact}, Amanda is the \textbf{author} and copyright
owner\footnote{It is assumed that copyright subsists in all of her works, as the
focus of the question is Raymond's infringement of said copyright.} for her
paintings, the photographs of her paintings and the published (online) editions
of the photographs of her paintings, as per [section
21(1)(a)]\cite{rsa78_copyrightact}.\\

\subsection{Exclusive Rights}
\label{sec:org4069f6a}

However, as per [section 7(a)]\cite{rsa78_copyrightact}, Raymond's printed
photographs of Amanda's paintings constitute infringing copies, that violate the
exclusive rights of the copyright owner, specifically in respect of the right to
either carry out or authorize the reproduction of the Amanda's
photographs. Moreover the right to make an adaptation of the work\footnote{Or
reproductions thereof.}, [section 7(e)]\cite{rsa78_copyrightact} vests solely with
the copyright owner. Raymond's copies constitute \emph{direct} or \emph{conscious} copies
Amanda's paintings, given that he had printed photographs of her original
paintings to work from.\\

\subsection{Direct Infringement}
\label{sec:org992c32b}
By performing any of the acts that a copyright vests the exclusive
right to either perform or authorize the performance thereof, without
the authors' consent constitutes a \emph{`direct infringement'}, [section
7]\cite{rsa78_copyrightact}.

\subsection{Indirect Infringement}
\label{sec:org283e75c}

Each time the \emph{`infringer'} Raymond, consciously furthers the commission of any
acts that only the authors are permitted to do or authorize to do, constitutes
an \emph{`indirect infringement'}. Such would be the case with the exploitation of
the infringing work, for example the sale or distribution [Section
23(2)(b-c)]\cite{rsa78_copyrightact}, or display in a public place (such as an art
gallery) of Raymond's adaptations (paintings) of the
printed photographs of Amanda's original paintings.

\subsection{Copyright Exceptions}
\label{sec:orge32431e}
It is important to note that if Raymond had used the printed copies of the
photographs of Amanda's paintings from her website, and his subsequent paintings
created from these copies, for his own personal or private use, research,
criticism or review, or for purposes of reporting current events, does not
constitute an unlawful act and copyright of Amanda's works would not have been
infringed upon, as per [section 15(4)]\cite{rsa78_copyrightact}. However given
that he produces the paintings under his own name, it is assumed that his usage
\uline{does not} constitute \emph{fair use}.\\

\subsection{Test for Copyright Infringement}
\label{sec:orgab2b5c7}
In establishing copyright infringement, it must be demonstrated that the
copyrighted work has indeed been copied, through the following two inquiries,
(\citetitle{corbett89_galago_v_erasmus}) \cite{corbett89_galago_v_erasmus}
\subsubsection{Objective Connection}
\label{sec:orgf91362f}
An objective connection between a substantial part of the copyright work and the
alleged infringing work - this follows clearly given that it is alleged that
Raymond printed copies of Amanda's photographs of her paintings, and based his
own paintings on those photographs. It may be possible for Raymond to argue that
the two works share a common prior art in which copyright does not subsist. This
would require a work which has substantial qualitatively similarities with both
Amanda's and Raymond's works.
\subsubsection{Subjection Connection}
\label{sec:org9e19583}
There should be a causal connection between the copyrighted work and the alleged
infringing work. It must be shown that Raymond had \emph{`access'} to the original
work - again this follows clearly as it is stipulated that he printed copies of
photographs of Amanda's completed paintings from her website.

\subsection{Yes, Raymond does infringe on the copyright in Amanda's work}
\label{sec:org77b19f1}
Even though Raymond's paintings themselves enjoy copyright protection as
an adaptation of an \textbf{artistic work}, as per [article
2]\cite{wipo86_berne}, regardless of whether their creation infringes on the
copyright of another work [section 2(3)]\cite{rsa78_copyrightact}; every time his work is sold, reproduced or distributed,
it will be an infringement of the copyright enjoyed by the Amanda's paintings,
photographs and published (online) work. Thus Raymond will be liable for
copyright infringement whenever he tries to exploit the work, as stipulated in [article 6]\cite{wto17_trips}.


\section{Leon is a young film maker from Cape Town.}
\label{sec:org13f58bb}

\subsection{Can Leon make commissioned films and still own the copyright in those films? [5]}
\label{sec:orgd493d1d}
As per [section 21(1)(c)]\cite{rsa78_copyrightact}, the owner of the copyright in
the commissioned cinematograph film will be the person who pays or agrees to pay
the monetary value for the film, as it is made in pursuance of said
commission.\\

However, this is subject to the provisions of [section 21(1)(b)]\cite{rsa78_copyrightact}, which stipulates that in the case where work
is commissioned for the purpose of publication, then the proprietor of the
organisation requesting the commission of the work will be the owner of the
copyright as far as it relates to publication or reproduction for the purposes
of publication, in all other respects however the author is the owner of any
copyright subsisting in the work.\\

It follows therefore based on the provisions of [section 21(1)(b)]\cite{rsa78_copyrightact}, one could
argue that the television channels commissioning the films will own the
broadcasting rights and any reproduction of the films for the purposes of them
being broadcast, but in all other respects Leon as the author, shall be owner of
any copyright subsisting in his films.

\subsection{Do exceptions [section 15(1) or 15(3)]\cite{rsa78_copyrightact} apply? [2]}
\label{sec:orgcf6e0ce}
Yes, [section 15(1)]\cite{rsa78_copyrightact} would apply given that the inclusion
of the bands performance is incidental and that the principle matter represented
in the film is the documentary and the guest being interviewed in said
restaurant.\\

[Section 15(3)]\cite{rsa78_copyrightact} would only apply in the case where the
band was a permanent fixture in the restaurant. It would be difficult to
reasonably argue that it was impossible to schedule the interview during a time
when the band was not playing at the restaurant. So yes this could apply but to
a lesser degree than [section 15(1)]\cite{rsa78_copyrightact}.

\subsection{Does the general fair dealing exception [section 12(1)]\cite{rsa78_copyrightact} apply? [8]}
\label{sec:org1bfc894}

\begin{itemize}
\item\relax [Section 12(1)(a)]\cite{rsa78_copyrightact} cannot apply as the production of
the documentary film is neither for the purposes or research, private study,
personal nor private use by Leon, but intended for commission in a film sold
to television channels.

\item\relax [Section 12(1)(b)]\cite{rsa78_copyrightact} cannot apply as the band's music is
playing in the background and incidental to the interview. Therefore one can
assume that the content of the interview is \uline{not} related to criticism of the
band's original music or that of another work that is juxtaposed against the
band's original work.

\item\relax [Section 12(1)(c)(i)]\cite{rsa78_copyrightact}, cannot apply as the
documentary will be produced as a cinematograph film, and not for publication
in a newspaper, magazine or similar periodical.
\end{itemize}

\subsubsection{Yes the general fair dealing exception (s 12(1)(c)(ii)) does apply}
\label{sec:org96cf70f}
It could be argued that as per [section 12(1)(c)(ii)]\cite{rsa78_copyrightact}, the
documentary and interview constitute a cinematograph film for the purposes of
reporting current events. Moreover as per the provisions of [section
12(c)]\cite{rsa78_copyrightact}, Leon is not required to mentioned the source, nor
the name of the band even if it appears in his cinematograph film. Based on this
he does not need to ask the bands permission, to include that scene in his film.

\section{Discuss the possibility that James can rely on the reverse-engineering defence [section 15(3A)]\cite{wipo96_copyright_treaty}. [10]}
\label{sec:orgf080bbf}
Given that Tania's factory both designs and manufactures the uniforms, as per [section (3A)(a)]\cite{rsa78_copyrightact},
it follows that three-dimensional reproductions (the physical school uniforms) where made from two-dimensional artistic
drawings or designs. This reproductions subsequently went on sale and were thus made available to the public with
Talia's consent.\\

As per [section (3A)(a)]\cite{rsa78_copyrightact}, it follows then that copyright has not been infringed when James,
without the consent of Talia, make available to the public three-dimensional reproductions or adaptations of the
authorized reproductions, provided that:
\begin{itemize}
\item the authorized reproductions primarily have a utilitarian purpose, as per [section
(3A)(a)(ii)]\cite{rsa78_copyrightact}. James could easily argue that school uniforms serve a utilitarian purpose.
\item the authorized reproductions are made by and industrial process, as per [section
(3A)(a)(ii)]\cite{rsa78_copyrightact}. Talia owns a factory that designs and manufactures the uniforms. James could
argue that the three-dimensional authorized reproductions of the uniforms design are manufactured via an industrial process.
\end{itemize}

\section{What moral rights are protected by \citetitle{rsa78_copyrightact}? [5]}
\label{sec:org2d9215c}
As per [section 20(1)]\cite{rsa78_copyrightact}, in spite of the transfer of the
copyright in a literary, musical or artistic work, cinematograph film or
computer program, the author may claim authorship of the work and oppose any
modifications, mutilations or distortions that would bring the author into
disrepute. However, an author who's authorised the use of his work in a
cinematograph film or television broadcast or computer program may not oppose
modifications necessary on a technical basis or for commercial exploitation.

\section{How my perceptions of copyright and/or specific copyright issues have changed during the course of this module. [15]}
\label{sec:org9e99675}
Whilst have an intuitive sense that the regulations regarding the implementation of copyright related issues would be
based on and resemble international instruments, I had no idea that the similarity and corroboration between said
instruments would be so strong.\\

With this idea in mind it follows quite naturally that the laws governing the protection of copyright should be
\emph{`universal'} in a sense, in as much as these international instruments should gives guidance and clarity to member
participants, but at the same time, they should provide enough freedom for individual member states to carefully articulate
and stipulate their \uline{own} legal and regulatory framework \cite{rsa78_copyrightact}, whilst working within the general scaffolding of the
international instruments, \cite{wipo83_paris_conve_protect_ip,wipo96_model_provi_unfair_comp,wipo96_copyright_treaty,wipo86_berne}.\\

Moreover, as a scientist, I particularly enjoyed how very specific aspects of the \textbf{theory} (framework of legal and
regulatory instruments) could be applied to described, predict and explain the experiment (actual legal case law and
hypothetical examples). Especially in instanced where legal precedent is yet to be established or where it is even
contradictory based on the \emph{interpretation} of one's viewpoint are arguments. In this instance a type of
phenomenological, or \emph{mathematically numerical approximation} is required that satisfies as many corroborating sources
as possible, whilst sufficiently reducing or mitigating the significance of the contradicting sources.
\printbibliography
\end{document}
