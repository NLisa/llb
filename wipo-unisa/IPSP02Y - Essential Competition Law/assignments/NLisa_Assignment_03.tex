% Created 2017-05-20 Sat 00:06
% Intended LaTeX compiler: pdflatex
\documentclass[11pt]{article}
\usepackage[utf8]{inputenc}
\usepackage[T1]{fontenc}
\usepackage{graphicx}
\usepackage{grffile}
\usepackage{longtable}
\usepackage{wrapfig}
\usepackage{rotating}
\usepackage[normalem]{ulem}
\usepackage{amsmath}
\usepackage{textcomp}
\usepackage{amssymb}
\usepackage{capt-of}
\usepackage{hyperref}
\usepackage[margin=0.80in]{geometry}
\usepackage[backend=bibtex, style=ieee]{biblatex}
\addbibresource{/home/nuk3/course/training/csir/novellasers/bibliography/bibliography.bib}
\DeclareFieldFormat[inproceedings]{citetitle}{\textit{#1}}
\DeclareFieldFormat[inproceedings]{title}{\textit{#1}}
\DeclareFieldFormat[misc]{citetitle}{#1}
\DeclareFieldFormat[misc]{title}{#1}
\renewcommand*{\bibpagespunct}{%
\ifentrytype{inproceedings}
{\addspace}
{\addcomma\space}}
\AtEveryCitekey{\ifuseauthor{}{\clearname{author}}}
\AtEveryBibitem{\ifuseauthor{}{\clearname{author}}}
\author{Nyameko Lisa}
\date{\today}
\title{IPSP02Y - Essential Competition Law\\\medskip
\large Assignment 3 - Exam}
\hypersetup{
 pdfauthor={Nyameko Lisa},
 pdftitle={IPSP02Y - Essential Competition Law},
 pdfkeywords={},
 pdfsubject={},
 pdfcreator={Emacs 25.2.1 (Org mode 9.0.7)},
 pdflang={English}}
\begin{document}

\maketitle
\section*{Declaration}
\label{sec:orge17480c}
\begin{itemize}
\item I know that plagiarism is to use someone else’s work and pass it off as my own.
\item I know that plagiarism is wrong.
\item I confirm that this assignment is my own work.
\item I have acknowledged in the bibliography accompanying the assignment all the sources that I have used.
\item I have not directly copied without acknowledgement anything from the Internet or from any other source.
\item I have indicated every quotation and citation in a footnote or bracket linked to that quotation.
\item I have not allowed anyone else to copy my work and to pass it off as their own work.
\item I understand that if any unacknowledged copying whatsoever appears in my assignment I will receive zero per cent for the assignment.
\item I am aware of the UNISA policy on plagiarism and understand that disciplinary proceedings can be instituted against me by UNISA if I contravene this policy.
\item I indicate my understanding and acceptance of this declaration by
entering my name hereunder:
\begin{itemize}
\item Name: \textbf{Nyameko Lisa} (Student Number: \textbf{7874-909-3})
\end{itemize}
\end{itemize}

\subsection*{NOTE}
\label{sec:org4b8cc11}
Please note that footnotes will be denoted as \footnote{This is a footnote.} and will
appear at the bottom of the page.\\
References will be denoted by \cite{wipo83_paris_conve_protect_ip} and will appear at the end of the document.
\newpage

\section{Identify three forms of unlawful competition which are applicable in the above scenario. [6]}
\label{sec:orgde4307c}
Brian may seek relief, amongst other grounds, on the basis of:
\begin{itemize}
\item \textbf{Passing off}, in light of the similarities of the registered trademarks both companies operate under, as per [article
10\(^{\text{bis}}\)(3)(1)]\cite{wipo83_paris_conve_protect_ip}, [article
2(1)]\cite{wipo96_model_provi_unfair_comp} and [article 16(1)]\cite{wto17_trips}.
\item \emph{Damage another's goodwill or reputation}, specifically the \textbf{Dilution of their
goodwill or reputation}, in regard to the lessening of distinctive character
or advertising value of the CHICKENBITE trademarks, as well as the appearance and
presentation of the product, as per [article 3(2)]\cite{wipo96_model_provi_unfair_comp}
\item \textbf{Unfair competition in respect of secret information}, where Danny unlawfully
acquired the memory stick with the recipe discovered by Brian, as per
[article 39(2)]\cite{wto17_trips} and [article 6(1)]\cite{wipo96_model_provi_unfair_comp}.
\end{itemize}

\section{Advise Brian if he can succeed with an action for unlawful competition against Danny. [18]}
\label{sec:org37e8747}
\subsection{Requirements for protection against unlawful competition based on Passing Off}
\label{sec:orgc611a82}
The plaintiff, Brian may argue that the conduct of the respondent, Danny,
amounts to unfair competition with respect to the trade name of his enterprise,
the registered of the goods he is selling (fried and grilled chicken), and is
trying to represent to the public that the respondent's enterprise and goods are
related to or affiliated with those of the plaintiff's. Intent of nor actual
confusion is required, simply the reasonable likelihood of such confusion
arising.\\

Brian may seek and interdict for the respondent to cease trading under his
registered trademark CHIC `N BITE and cease producing goods that are similar to
the plaintiff's, if he can prove:
\begin{itemize}
\item CHICKEN BITE *has acquired with the public a reputation associated
with his goods, services and or business. This follows naturally given that
his outlets are well-known across South Africa and considered to produce tasty
grilled and fried chicken.
\item CHIC `N BITE \textbf{conduct is like to deceive or confuse the public}. This also
follows naturally given that the names are so similar and that he directly
copied the spice mixture.
\end{itemize}

There exist a number of examples in the context of South African case law that have
qualified for the protection against unlawful competition based on passing
off. \emph{In the interests of non-repetition, kindly refer to the following section for detailed descriptions of said examples.}

\subsection{Requirements for protection against unlawful competition based on Damage (Dilution) of Goodwill or Reputation}
\label{sec:org73c94bf}
The plaintiff, Brian may argue that the respondent Danny, acted unlawfully with
respect to a concealed misappropriation within a competitive context and can
apply to seek an interdict with respect to the defendant's enterprise
to cease to trade with under their trade name and trade mark. To succeed in
such an action the plaintiff must present the following:
\begin{itemize}
\item CHICKEN BITE outlets are well-known across South Africa, the trade name and
trademark CHICKEN BITE has indeed \textbf{acquired and reputation and advertising
value} in terms of the enterprises tasty grilled and fried chicken.
\item The respondent, Danny, in establishing CHIC `N BITE, used \textbf{without the consent}
of the plaintiff, a \textbf{name and / or trademark} very similar to the plaintiff's,
in relation to the enterprise, goods and services or the respondent. The
plaintiff did not authorise the respondent to make use of the information on
his memory stick.
\item Through the registering of the trademark CHIC `N BITE and the use of the
secret spices in establishing his enterprise, the respondent has impaired the
plaintiff's goodwill or at least threatens to impair the goodwill of the
plaintiff name and / or mark, through dilution of the advertising value and /
or distinctive character of the plaintiff's name or mark.
\end{itemize}

The plaintiff should be advised that this will likely treated as an action for
unlawful competition based on Passing Off as described above. No reported South
African decision has seen relief granted based on this form of unfair
competition.\\

For example \citetitle{joffe94_fifa_bartlett} \cite{joffe94_fifa_bartlett}. This
case involved concealed misappropriation between related enterprises,
specifically the organizer of the soccer
World Cup and respondents misrepresenting that they were holder's of the Soccer
World Cup USA 1994, licensing rights in South Africa. The court held that
character merchandising was prevalent in South Africa and that the advertising
values of the associated names, characters or insignia in the products used
could be exploit to boost sales. The court granted an interdict on the basis of
passing off, likely to cause injury or damage to the goodwill of the plaintiff
as the respondents had misrepresented that they were holders of the World Cup
licensing rights in South Africa.\\

Another example of concealed misappropriation between related enterprises
concerned \citetitle{holmes77_capital_v_holiday}
\cite{holmes77_capital_v_holiday}. In which the plaintiff \emph{Holiday Inn} applied
for an interdict to restrain the respondent from misusing the distinctive name to
cause confusion and misappropriate the advertising value in respect of a name
for the respondent's shopping centre and duplex apartments. The court granted an
interdict on the basis of passing off.
\subsection{Requirements for protection against Unfair Competition in respect of secret information}
\label{sec:org304851b}

The plaintiff must show that the information is:
\begin{itemize}
\item \textbf{Confidential:} The spices used to prepare the plaintiff's chicken was
\uline{discovered} by the plaintiff after he visited India, Brazil and
Portugal. This information was neither public knowledge nor released in the
public domain. The respondent only gained access to this knowledge from the
plaintiff's flash drive, which he may also allege was stolen by the respondent
in a combined matter.
\item Has \textbf{trade value}: The respondent used the same mixture of ingredients to
manufacture his own spices for his chicken. Given that the plaintiff's outlets
are well known for their tasty grilled and fried chicken, if follows naturally
that the confidential information of the plaintiff has significant trade value
to the respondent as he is a rival competitor.
\end{itemize}

There are a number of examples in South African case law that have qualified for
the protection of secret information, specifically in regards to a manufacturing
process \emph{for example Harvey Tiling Co (Pty) Ltd v Rodomac (Pty) Ltd $\backslash$& Another
1977 (1) SA 316 (T)}. Another example that was successfully tried as a matter
for the protection of secret information, was
\citetitle{dijkhorst81_atlas_v_pikkewyn}, \cite{dijkhorst81_atlas_v_pikkewyn}, where
judge Dijkhorst was famously quoted that this branch of law was designed to
\emph{``address the schemes of geniuses bent upon reaping what they have not sown.''}
One last successful example which qualified for the protection of secret
information was \citetitle{diemont72_stellwinetrust_v_oudemeester}
\cite{diemont72_stellwinetrust_v_oudemeester}, where Judge Diemont stipulated that
one is acting unlawfully and dishonestly when he `filches' information from a
competitor, devised through the skill and industry of that competitor for his
own profit.

\section{Discuss whether the approach followed by South African courts / law complies with the requirements imposed by international conventions and other instruments. [26]}
\label{sec:org0cdf5df}
\subsection{Passing off}
\label{sec:orgf3cfaaf}
Due to its significance and importance, this particular [article
10\(^{\text{bis}}\)(3)(1)]\cite{wipo83_paris_conve_protect_ip}, [article
2(1)]\cite{wipo96_model_provi_unfair_comp} and [article 16(1)]\cite{wto17_trips},
all make reference to it. In particular  the Model Provisions stimulates that /‘Any act or practice, in the course of industrial or commercial
activities, that causes, or is likely to cause, confusion with
respect to another’s enterprise or its activities, in particular,
the products or services offered by such enterprise, shall
constitute an act of unfair competition. ’/ Moreover neither actual confusion nor the intent to confuse is required as per
[article 2(2)]\cite{wipo96_model_provi_unfair_comp} where particular articulation
of a trademark, tradename or appearance of a product is specified. CHICK 'N BITE
has introduced a false affiliation between themselves and the plaintiff. These considerations presented in the international instruments are in are
agreement with South African case law.

\printbibliography
\end{document}
